\message{ !name(chap04.tex)}
\message{ !name(chap04.tex) !offset(-2) }

\chapter{算法的设计与实现}
\label{chap:algo}

\section{设计}
\label{sec:design}
模拟退火算法是一个可以被高度抽象的算法。它的设计并不受制于所需要求解的问题(但是参数的设置需要根据问题的性质来设定)。我们设计的算法主要框架如下:

\begin{algorithm}[H]
  \DontPrintSemicolon
  \SetKwInOut{Input}{INPUT}
  \SetKwInOut{Output}{OUTPUT}
  \SetKwData{Temp}{temperature}
  \SetKwData{CurS}{cur\_state}
  \SetKwData{CurV}{cur\_value}
  \SetKwData{Neigh}{neighbor}
  \SetKwData{NewV}{new\_value}
  \SetKwData{Term}{term}
  \SetKwFunction{GetV}{value}
  \SetKwFunction{GetNeigh}{get\_neighbor}
  \SetKwFunction{Acc}{accept}
  \SetKwFunction{Cool}{cooldown}
  
  \Input{List $si$ which is a permutation of ids of PMs}
  \Output{List $so$ which is another permutation of ids of PMs}
  \Begin{
    \CurS$\leftarrow si$\;
    \CurV$\leftarrow$ \GetV{$si$}\;
    \Temp$\leftarrow$ $InitTemperature$\;
    \Term$\leftarrow$ $0$\;
    \While{$\Term < maxTerm \And \Temp > minTemperature $}
    {
      \Neigh $\leftarrow$ \GetNeigh{\CurS}\;
      \NewV $\leftarrow$ \GetV{\Neigh}\;
      \If{\Acc{\CurV,\NewV,\Temp}}
      {
        \CurS $\leftarrow$ \Neigh\;
        \CurV $\leftarrow$ \NewV\;
      }
      $\Term \leftarrow \Term + 1$\;
      \Temp $\leftarrow$ \Cool{\Temp,\Term}\;
    }
    \Return \CurS
  }
  \caption{Simulated Annealing\label{SA}}
\end{algorithm}

算法的输入是一个``状态''$S$,表示形式为一个列表,表中的元素$S_i$表示$VM_i$被部署到$PM_{S_i}$。一个状态即代表问题的一个解。

算法的输出童谣是一个状态,这是经过模拟退火算法迭代过后的新状态,它收敛到一个局部最优解。

\subsection{初始化}
\label{sec:init}

由于算法对初始状态并不敏感,因此我们采用了\texttt{random\_fit}的方法生成初始状态,其算法描述如下:

\begin{algorithm}[H]
  \DontPrintSemicolon
  \SetKwInOut{Input}{INPUT}
  \SetKwInOut{Output}{OUTPUT}
  \SetKwData{St}{state}
  \SetKwData{Pl}{plist}
  \SetKwData{PIndex}{pi}
  \SetKwData{PData}{p}
  \SetKwFunction{Random}{randomInt}
  \SetKwFunction{Avail}{available}
  \SetKwFunction{Assign}{assign}
  \SetKwFunction{Len}{length}
  
  \Input{List $V$ which is the VMs\' requests infomation}
  \Output{List $s$ which is a state(solution)}
  \Begin{
    \St $\leftarrow$ [~]\;
    \Pl $\leftarrow EmptyPmInfoList$ \;
    \For{$every vm_i \in V$}
    {
      \PIndex $\leftarrow$ \Random{0,\Len{\Pl}}\;
      \PData $\leftarrow$ \Pl[\PIndex]\;
      \Repeat{$available(vm_i,\PData)$}{
        \PIndex $\leftarrow$ \Random{0,\Len{\Pl}}\;
        \PData $\leftarrow$ \Pl[\PIndex]\;
      }
      \Assign{$vm_i$,\PData}\;
      \St[$i$] $\leftarrow$ \PIndex\;
    }
    \Return \St\;
  }
  \caption{Random Fit\label{rf}}
\end{algorithm}

算法\ref{rf}随机的在提供的PM序列里寻找可以(满足约束条件)部署的节点。直到所有VM均被成功部署后返回解状态。

\section{实现}
\label{sec:implementation}

我们使用了Python编程语言对算法进行了实现。下面介绍一些值得注意的部分。

\subsection{参数的设置}
\label{sec:config}

模拟退火算法对参数设置是非常敏感的,包括初始温度的设置,降温的速度,评价函数的设计都会对最后收敛的速度和结果有比较大的影响。经过反复试验,我们确定了以下的参数设置:

\begin{itemize}
\item \textbf{初始温度}。由于在判断是否接受时计算了$\Delta value/t$的值,因此我们希望能将温度和评价归一化。初始温度被设置为初始评价的$K$倍,$K = N_v N_p$。
\item \textbf{降温速度}。降温速度太快会导致算法很快收敛,得不到比较理想的结果,而降温速度太慢则会导致收敛过程很长,并且在计算初期接受过于不理想的结果,影响效率。经过试验,使用了$t_{n+1} = 0.89t_n$这样的线形降温速度,收到了良好的效果。
\item \textbf{评价函数}。在实现中采用了四种因素综合评价的方式。我们考虑了PM的数量,RAM的负载均衡,CPU的负载均和和平均冲突率,并以这四个值的乘积作为评价函数的返回值。\label{evalutation}
\end{itemize}

\subsection{评价函数}
\label{sec:evaluation-function}

在所有参数中,评价函数的确定是对算法影响最大的,因为它直接决定了算法的走向。比如我们选取使用PM的数量作为评价的标准,那么在迭代的过程中接受那些使用更少PM的解。如前文\ref{evaluation}所说,我们希望综合考虑更重因素,所以我们采用了如下的评价函数:

\begin{itemize}
\item 
\end{itemize}
\message{ !name(chap04.tex) !offset(-119) }
