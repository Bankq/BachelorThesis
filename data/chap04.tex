
\chapter{算法的设计与实现}
\label{chap:algo}

\section{设计}
\label{sec:design}
模拟退火算法是一个可以被高度抽象的算法。它的设计并不受制于所需要求解的问题(但是参数的设置需要根据问题的性质来设定)。我们设计的算法主要框架如下:

\begin{algorithm}[H]
  \DontPrintSemicolon
  \SetKwInOut{Input}{INPUT}
  \SetKwInOut{Output}{OUTPUT}
  \SetKwData{Temp}{temperature}
  \SetKwData{CurS}{cur\_state}
  \SetKwData{CurV}{cur\_value}
  \SetKwData{Neigh}{neighbor}
  \SetKwData{NewV}{new\_value}
  \SetKwData{Term}{term}
  \SetKwFunction{GetV}{value}
  \SetKwFunction{GetNeigh}{get\_neighbor}
  \SetKwFunction{Acc}{accept}
  \SetKwFunction{Cool}{cooldown}
  
  \Input{List $si$ which is a permutation of ids of PMs}
  \Output{List $so$ which is another permutation of ids of PMs}
  \Begin{
    \CurS$\leftarrow si$\;
    \CurV$\leftarrow$ \GetV{$si$}\;
    \Temp$\leftarrow$ $InitTemperature$\;
    \Term$\leftarrow$ $0$\;
    \While{$\Term < maxTerm \And \Temp > minTemperature $}
    {
      \Neigh $\leftarrow$ \GetNeigh{\CurS}\;
      \NewV $\leftarrow$ \GetV{\Neigh}\;
      \If{\Acc{\CurV,\NewV,\Temp}}
      {
        \CurS $\leftarrow$ \Neigh\;
        \CurV $\leftarrow$ \NewV\;
      }
      $\Term \leftarrow \Term + 1$\;
      \Temp $\leftarrow$ \Cool{\Temp,\Term}\;
    }
    \Return \CurS
  }
  \caption{Simulated Annealing\label{algo:SA}}
\end{algorithm}

算法的输入是一个``状态''$S$,表示形式为一个列表,表中的元素$S_i$表示$VM_i$被部署到$PM_{S_i}$。一个状态即代表问题的一个解。

算法的输出同样是一个状态,这是经过模拟退火算法迭代过后的新状态,它收敛到一个局部最优解。

\subsection{初始化}
\label{sec:init}

由于算法对初始状态并不敏感,因此我们采用了\texttt{random\_fit}的方法生成初始状态,其算法描述如下:

\begin{algorithm}[H]
  \DontPrintSemicolon
  \SetKwInOut{Input}{INPUT}
  \SetKwInOut{Output}{OUTPUT}
  \SetKwData{St}{state}
  \SetKwData{Pl}{plist}
  \SetKwData{PIndex}{pi}
  \SetKwData{PData}{p}
  \SetKwFunction{Random}{randomInt}
  \SetKwFunction{Avail}{available}
  \SetKwFunction{Assign}{assign}
  \SetKwFunction{Len}{length}
  
  \Input{List $V$ which is the VMs\' requests infomation}
  \Output{List $s$ which is a state(solution)}
  \Begin{
    \St $\leftarrow$ [~]\;
    \Pl $\leftarrow EmptyPmInfoList$ \;
    \For{$every vm_i \in V$}
    {
      \PIndex $\leftarrow$ \Random{0,\Len{\Pl}}\;
      \PData $\leftarrow$ \Pl[\PIndex]\;
      \Repeat{$available(vm_i,\PData)$}{
        \PIndex $\leftarrow$ \Random{0,\Len{\Pl}}\;
        \PData $\leftarrow$ \Pl[\PIndex]\;
      }
      \Assign{$vm_i$,\PData}\;
      \St[$i$] $\leftarrow$ \PIndex\;
    }
    \Return \St\;
  }
  \caption{Random Fit\label{algo:rf}}
\end{algorithm}

算法\ref{algo:rf}随机的在提供的PM序列里寻找可以(满足约束条件)部署的节点。直到所有VM均被成功部署后返回解状态。

\subsection{找寻邻解}
\label{sec:neighbor-algo}
算法\ref{algo:SA}中使用的\texttt{get\_neighbor(state)}过程用来从当前状态生成一个相邻的状态,找寻邻解的算法描述如下:

\begin{algorithm}[H]
  \DontPrintSemicolon
  \SetKwInOut{Input}{INPUT}
  \SetKwInOut{Output}{OUTPUT}
  \SetKwData{St}{state}
  \SetKwData{NewSt}{neighbor}
  \SetKwData{Src}{src}
  \SetKwData{Des}{des}
  \SetKwData{Pl}{plist}
  \SetKwData{PIndex}{pi}
  \SetKwData{PData}{p}
  \SetKwData{VData}{v}
  \SetKwData{Max}{max}
  \SetKwFunction{Random}{randomInt}
  \SetKwFunction{Avail}{available}
  \SetKwFunction{Assign}{assign}
  \SetKwFunction{Len}{length}
  
  \Input{Current state $s$}
  \Output{Neighbor state $neighbor$}
  \Begin{
    \NewSt $\leftarrow$ \St\;
    \For{$0$ \KwTo \Max }
    {
      \Src $\leftarrow$ \Random{0,$N_v$} \;
      \Des $\leftarrow$ \Random{0,$N_p$} \;
      \VData $\leftarrow VM_{\Src}$\;
      \PData $\leftarrow PM_{\Des}$\;
      \Repeat{\Avail{\VData,\PData}}
      {
        \Des $\leftarrow$ \Random{0,$N_p$} \;
        \PData $\leftarrow PM_{\Des}$\;
      }
      \NewSt[~\Src] $\leftarrow$ \Des\;
    }
    \Return \NewSt\;
  }
  \caption{Get Neighbor\label{algo:neighbor}}
\end{algorithm}

在算法\ref{algo:neighbor}中,我们对当前状态中的$max$个值进行了重新随机部署,$max$的取值决定了生成的邻解和当前解的距离。在模拟退火算法中,我们希望邻解有较高的随机性但是与当前解的距离不能太远,因此在实现中通常选$\frac{1}{10}$比例的值。

\subsection{接受条件}
\label{sec:accept}
算法\ref{algo:SA}中使用的\texttt{accept(cur,new,t)}用来判断是否接受当前邻点。根据Metropolis准则,金属粒子在温度T时趋于平衡的概率为$e^{-\frac{\Delta E}{KT}}$,其中K是Boltzmann常数。本文所使用的策略即来自这一准则,算法的描述如下:
\begin{algorithm}[H]
  \DontPrintSemicolon
  \SetKwInOut{Input}{INPUT}
  \SetKwInOut{Output}{OUTPUT}
  \SetKwData{CV}{cur}
  \SetKwData{NV}{new}
  \SetKwData{Del}{delta}
  \SetKwData{Tem}{t}
  \SetKwData{True}{true}
  \SetKwData{False}{false}
  \SetKwFunction{Random}{random}
  
  \Input{Current value \CV, neighbor value \NV, current temperature \Tem }
  \Output{Boolean value indicates accept neighbor or not}
  \Begin{
    \Del $\leftarrow \NV - \CV$\;
    \If{$\Del < 0$}{
      \Return \True\;
    }
    \Else
    {
      \If{$e^{-\frac{\Del}{t}} > \Random{0,1}$}
      {
        \Return \True\;
      }
    }
    \Return \False
  }
  \caption{Accept\label{algo:accept}}
\end{algorithm}
在上述算法\ref{algo:accept}中,\texttt{random(a,b)}函数随机返回一$a$到$b$之间的实数。

\section{实现}
\label{sec:implementation}

我们使用了Python编程语言对算法进行了实现。下面介绍一些值得注意的部分。

\subsection{参数的设置}
\label{sec:config}

模拟退火算法对参数设置是非常敏感的,包括初始温度的设置,降温的速度,评价函数的设计都会对最后收敛的速度和结果有比较大的影响。经过反复试验,我们确定了以下的参数设置:

\begin{itemize}
\item \textbf{初始温度}。由于在判断是否接受时计算了$\Delta value/t$的值,因此我们希望能将温度和评价归一化。初始温度被设置为初始评价的$K$倍,$K = N_v N_p$。
\item \textbf{降温速度}。降温速度太快会导致算法很快收敛,得不到比较理想的结果,而降温速度太慢则会导致收敛过程很长,并且在计算初期接受过于不理想的结果,影响效率。经过试验,使用了$t_{n+1} = 0.89t_n$这样的线形降温速度,收到了良好的效果。
\item \textbf{评价函数}。在实现中采用了四种因素综合评价的方式。我们考虑了PM的数量,RAM的负载均衡,CPU的负载均衡和平均冲突率,并以这四个值的乘积作为评价函数的返回值。\label{evalutation}
\end{itemize}

\subsection{评价函数}
\label{sec:evaluation-function}

在所有参数中,评价函数的确定是对算法影响最大的,因为它直接决定了算法的走向。比如我们选取使用PM的数量作为评价的标准,那么在迭代的过程中接受那些使用更少PM的解。如前文\ref{evaluation}所说,我们希望综合考虑更重因素,所以我们采用了如下的评价函数:

\begin{itemize}
\item \textbf{PM的数量}。我们用\texttt{count\_active(plist)}函数来评价。函数接受一个根据状态生成的PM信息列表,计算其中被``激活''(有VM部属在其上)的PM的个数(为了不同规模数据间的比较,实际返回的是PM的个数与$N_v$的比例。
\item \textbf{RAM的均衡负载}。我们用\texttt{ram\_stdev(plist)}函数来评价。函数接受一个根据状态生成的PM信息列表,计算RAM利用率的均方差(Standard Deviation)。如果RAM利用率的均方差的足够小则说明每台PM上的RAM负载比较均衡。
\item \textbf{CPU的均衡负载}。和RAM的均衡负载类似,我们用\texttt{cpu\_stdev(plist)}函数来评价。函数接受一个根据状态生成的PM信息列表,计算CPU利用率的均方差。如果RAM利用率的均方差的足够小则说明每台PM上的CPU负载比较均衡。
\item \textbf{平均冲突率}。我们定义平均冲突率$\mu \geq 1$,使用\texttt{count\_conficts(plist)}函数来计算。函数接受一个根据状态生成的PM信息列表,结算其中RAM利用率超过$1$的PM的平均RAM利用率。如果不存在利用率超过$1$的PM,则取值$1$.
\end{itemize}

\subsection{测试数据的选取}
\label{sec:data}
通常,云计算服务提供商提供的用户的选择都是有限的,比如提供给用户选择``双核500MB内存'',通过限制虚拟机请求的数据可以针对性的优化部属算法。在这种情境下,\texttt{First-Fit}算法通常能收到良好的效果。

在本文所描述的情境中,为了更好的模拟不同情境下算法的效果,我们没有对请求的数据进行限制(但是为了避免无解的情况,对数据的上下限进行了规定)。我们尝试了多种测试数据生成的方法,最终确定以下的测试数据:
\begin{itemize}
\item 每组测试设计都是\textbf{随机}生成的。
\item 将$C^T$设置为$8$,即认为每个实体机都具有$8$个计算核心;将$R_T$设为$1$。
\item $C^{req}_i$将取$0.1-8.0$之间的随机数;$R^{req}_i$取$0.2-0.8$之间的随机数;$ER_i$取$0.2-0.8$之间的随机数。生成的数据可以认为即包含运算需求高的请求也包含存储需求高的请求。
\item 测试数据的数量取$100-1000$间隔$100$的整数,用以模拟不同规模下的请求。
\end{itemize}

在随机的数据和不同的数据规模上的得到相似测试结果可以从某种程度上证明算法的稳定性和可行性。
