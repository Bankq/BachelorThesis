
\chapter{相关工作}
\label{chap:2}

\section*{VBP问题}
一维装箱问题已经被深入研究。Fernandez de la Vega和Lueker\cite{Vega81}给出了首个线形时间逼近策略(APTAS)。他们的算法随后被Karmarker和Karp\cite{Karm82}改进,达到了$(1+log^2)$-\textsf{OPT}上界。

对于二维装箱问题,Woeginger\cite{Woeg97}证明了不存在APTAS。对于更高维的情况,Fernandez de la Vega和Lueker拓展了一维时的算法,提出了一种$(d+\epsilon)$-\textsf{OPT}的算法。Chekuri和Khanna\cite{Chan99}提出了一个$O(log~d)$近似算法,对于给定的$d$,算法运行在线形时间复杂度内。Bansal等\cite{Bans07}改进了这个结果,提出了一个对于任意$\epsilon \geq 0$的$(ln~d + 1 + \epsilon)$逼近算法。Karger等\cite{Karg07}提出了一种对于多维随机样例的VBP问题使用技术的线形逼近方法。

\section*{虚拟机部署问题}

虚拟机部署问题是整个云计算的核心问题,有很多研究都显示了合理部署虚拟机的重要性\cite{Grit06}\cite{Bian04}。许多基于First Fit Decrease(FFD)的算法被应用在虚拟机部署问题之中。Verma等\cite{Verm08}提出了一个能够通过尽量减少迁移\footnote{迁移是指对已经部署的虚拟机进行在其他实体机上再次部署的操作}达到最佳部署的方案。Hyser等\cite{Hyse07}提出了一个互动式的再部署方案,用以解决动态情境\footnote{动态情境指在算法开始前不能完全得知虚拟机请求的信息,需要根据实时信息进行部署的问题}下的问题。Bobroff等\cite{Bobr07}提出了一种预测请求并部署的动态规划算法。Shahabuddin等\cite{Shah01}提出了一种简单的启发式算法。

即使现阶段业界纷纷趋向虚拟化技术,虚拟机部署问题仍研究尚浅。由于该问题的本质是一个NP-Hard问题,寻求最佳解效率比较低,为了客服这个问题,寻找一种效率较高并仍能获得比较好的结果的方法,我们提出了一个基于模拟退火算法,并能适应不同部署需求的算法。

\subsection*{模拟退火算法}
\label{sec:sa-inro}

模拟退火算法是一种基于概率的启发式算法,通常用来求解组合优化问题,寻找全局最优解。对于特定的问题,模拟退火的效率通常会优于全局搜索。Bohachevsky等证明,模拟退火算法依概率收敛到全局最优点。

模拟退火算法的名称来源冶金学专有的名词退火。退火是将材料加热后再以特定速率冷却,目的是增大结晶体的体积。我们将热力学的理论模拟至统计学上,将解空间内每一点相像成空气中的分子;分子的能量是它本身的动能,而解空间的每一点也像分子一样带有能量,以表示对命题的合适程度。演算法先以搜寻空间内任意一点作起始,每一步先选择一个``邻居'',然后再计算从现有位置到达``邻居''的概率。

\subsection*{启发式算法适合虚拟机部署问题}
\label{sec:advantage-of-sa}

对于虚拟机部署问题,一个解可以被认为是单个VM状态的组合,算法的目标是对组合进行优化,寻找全局最优解。解的优劣可以用评价函数来体现。解的维度是虚拟机请求的规模,但是对于大规模的虚拟机请求,解空间将变的巨大,经典的确定性算法将耗费巨大的计算时间。使用启发式算法,特别是基于概率的模拟退货算法,可以有效的在巨大的解空间中快速逼近全局最优点。

本文剩余部分将以下方式组织:第三部分将对本文所希望解决的问题进行描述和形式化;第四部分将给出算法的设计思路和实现,并且进行模拟实验得出结果;第五部分将对产生的结果进行分析,并给出算法的评价。







