\chapter{总结与展望}
以提高资源利用率为指导思想,本文进行了以下几方面的研究。首先,通过分析物理资源(CPU,内存)资源的使用特点,我们在传统的虚拟机部署问题背景下引入内存资源共享机制。通过多VM进行内存资源的分时复用,可以有效的减少资源的闲置,提高资源的有效利用率。其次,针对共享内存可能引起的冲突问题,通过资源的使用概率来对可能产生的冲突进行描述,而所采用的模拟退火算法能够使冲突的产生降至较小的水平,而通过参数的设定可以杜绝高冲突情况的产生。最后,通过充分的实验验证,和经典的First-Fit算法进行对比,我们所提出的基于模拟退火算法在负载均衡和使用的PM数量上都有较大优势。

对资源使用的冲突问题是解决该部署问题的关键所在,对冲突的刻画和避免仍有很大的改进空间。首先,相比于目前采用的单一概率模式,资源的使用情况通过使用量的概率分布来描述更加合理。这也带来了多了VM使用资源的联合概率分布的描述等问题,需要进一步的后续研究。其次,冲突无可避免的会发生,如何妥善处理冲突发生时各VM资源的分配,也是一个需要进一步研究的问题。

提高资源利用率是本文的核心思想,云计算环境下,尤其是大数据中心背景下,存在着比较严重的资源浪费现象。开展该项研究以各种突进提高资源利用率进而减少能源等开销是有意义的,也将继续成为一个重要的研究方向。