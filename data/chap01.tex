
\chapter{引言}
\label{chap:1}

\section{研究背景}
\label{sec:background}

随着信息技术的广泛应用和快速发展,云计算作为一种新兴的商业计算模型得到了人们的广泛关注。简单说,云计算是一种基于互联网的计算方式,通过这种方式,包括内存,计算能力(CPU资源),带宽等资源可以分配给有需求的用户。互联网上的云计算服务特征和自然界的云、水循环具有一定的相似性,因此,云是一个相当贴切的比喻。通常云计算服务应该具备以下几条特征:
\begin{itemize}
\item 基于虚拟化技术快速部署资源或获得服务
\item 实现动态的、可伸缩的扩展
\item 按需求提供资源、按使用量付费
\item 通过互联网提供、面向海量信息处理
\end{itemize}

云计算可以分为``软件即服务''(SaaS)、``平台即服务''(PaaS)、``架构即服务''(IaaS)。这些服务通过虚拟化技术提供给用户。目前比较有名的相关服务有Amazon EC2,GoGrid,Windows Azure和Rackspace Cloud。

虚拟化技术的引入为实现云计算的弹性资源供给、按需服务的等特性提供了保障。在云计算中使用的虚拟化技术通常被称为平台虚拟化技术,通过使用控制程序,隐藏特定计算平台的实际物理特性,为用户提供抽象的、统一的、模拟的虚拟环境(称为虚拟机)。在云计算服务提供商提供给用户的服务是以部署在实体机(Physical Machine,PM)之上的虚拟机(Virtual Machine,VM)实现的,通常的实现形式是将VM部署在大规模的计算集群中。在一台PM上可能存在多个VM。因此,我们希望能够以最优的方式调度计算资源,从而尽可能的减少使用PM的数量,达到节省成本(主要是能源成本)、提高服务质量的目的。

如何高效的部署VM,并尽可能利用已有资源是一个重要的问题。在部署VM时,需要考虑的资源有实体机的内存、带宽、计算能力(CPU)等因素。这个问题可以被视为一个多维向量装箱问题\footnote{多维装箱问题的定义是: {\kai 给定一个有$n$个元素的$d$维向量的集合$S$,$S = \{\thinspace p_1,p_2,\cdots,p_n\ |\ p_i \in [0,1]^d \}$,寻找一个S上的划分$A_1,A_2,\cdots,A_m$使得$\sum_{p \in A_i}p^k\leq 1,\forall i,k$}}(multi-dimensional vector bin packing problem,已被证明为一个NP-Complete问题\cite{Garey76})。VM所需要的资源被视为一个d维向量,其中每一维都是一个非负的值(即``小球'',一般取$(0,1)$的值);而每个PM所拥有资源也可以被看作一个d维向量,其中每一维,和VM的资源请求一样,代表一个独立的资源(即``箱子'',一般取值为$1$)。我们的目标是尽可能的减少``箱子''的数量并且能够满足将所有的VM请求部署在PM资源上且保证每个PM的上某一维上的请求都没有超过``箱子''的容量。因此,资源分配问题就可以视为一个d维装箱问题。



\section{研究动机及本文工作}
\label{sec:concern}


在很多企业提供的VPS(Virtual Private Server,虚拟专用服务器)服务中,提供给用户的选择很多都是不同的处理器性能和内存的大小。在实际的VM部署中,除了资源分配的算法问题之外,还需要考虑如网络拓扑设计等许多方面的问题。但其中最关键也是最核心的问题还是如何根据用户的请求部署、移动VM。在VM部署考虑的资源中,计算资源和内部存储资源是最关键的,而外部存储资源(硬盘)等由于其成本相对较低,使用相对不频繁,因此通常并不做主要因素。

PM资源利用率直接关系到云平台的工作效率以及运维成本,本文以提高PM资源有效利用率为目标,采用资源复用的思想,使用模拟退火算法解决因复用而产生的冲突问题,并通过实验表明了复用机制的有效性。具体而言,本文工作包括几下几个方面:
\begin{enumerate}
\item \textbf{分析资源使用特点,引入内存资源复用机制。}\\
根据CPU使用的原子性以及内存资源的可共享性,我们对提出的VM请求进行分别处理:独占式的满足CPU请求,以概率共享的方式满足内存资源请求。内存共享机制可以减少内存资源的空置率,从而提高整体资源利用率。
\item \textbf{通过概率模式刻画使用冲突并采取模拟退火算法规避冲突。}\\
由于概率共享方式的引入,在多VM共同使用内存时必然会引起VM间对内存使用的冲突。我们采用概率的方式来刻画冲突的强弱程度,并在部署时使得冲突的产生控制在合理的概率范围内。我们所使用的基于模拟退火的算法能够很好的做到这一点。
\item \textbf{进行大量实验,给出了资源复用机制在提高资源利用率方面的有效性}\\
针对该问题背景,我们提出了考虑所使用的PM数量、负载均衡、冲突概率等多因素在内的效果评价体系。通过大量的模拟实验,我们发现:在牺牲少量服务质量(由冲突引起)的前提下,引入内存共享机制可以有效的提高整体资源利用率。另一方面,实验结果也表明了模拟退火算法在控制冲突方面有着优异的表现。
\end{enumerate}

\section{本文的组织结构}
\label{structure}


本文的剩余部分将按以下方式组织:第二部分将对相关工作进行介绍;第三部分将对本文所希望解决的问题进行分析以及形式化描述并介绍算法的设计;第四部分将对设计的算法进行实现,模拟实验得出结果并对产生的结果进行分析,并给出算法的评价;最后一部分进行总结并展望未来工作。



 


 



