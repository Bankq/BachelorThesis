
\chapter{引言}
\label{chap:1}

\section{云计算}
\label{sec:cloud-computing}

云计算(Cloud Computing)是这两年非``火热''一个概念。简单说,云计算是一种基于互联网的计算方式,通过这种方式,包括内存,计算能力(CPU资源),带宽等资源可以分配给有需求的用户。互联网上的云计算服务特征和自然界的云、水循环具有一定的相似性,因此,云是一个相当贴切的比喻。通常云计算服务应该具备以下几条特征:
\begin{itemize}
\item 基于虚拟化技术快速部署资源或获得服务
\item 实现动态的、可伸缩的扩展
\item 按需求提供资源、按使用量付费
\item 通过互联网提供、面向海量信息处理
\item 用户可以方便地参与
\item 形态灵活,聚散自如
\item 减少用户终端的处理负担
\item 降低了用户对于IT专业知识的依赖
\end{itemize}

云计算环境可以分为``软件即服务''(SaaS)、``平台即服务''(PaaS)、``架构即服务''(IaaS)。这些服务通过虚拟化技术提供给用户。目前比较有名的相关服务有Amazon EC2,GoGrid,Windows Azure和Rackspace Cloud。

\section{虚拟资源的部署}
\label{sec:virtual-machine-placement}

在云计算中,用户使用的是部署在一系列实体机之上的虚拟机,通常的实现形式是将虚拟机部署在大规模的计算集群中。在一台实体机(PM,下同)上可能存在多个虚拟机。因此,我们希望能够以最优的方式部署计算资源,从而尽可能的减少使用实体机的数量,达到节省成本(主要是能源成本)的目的。

在部署虚拟机湿,需要考虑的资源有实体机的内存、带宽、计算能力(CPU)等因素。这个问题可以被视为一个多维装箱问题(multi-dimensional vector bin packing problem)。VM所需要的资源被视为一个d维向量,其中每一维都是一个非负的值(即``小球'',一般取$0-1$的值);而每个PM所拥有资源也可以被看作一个d维向量,其中每一维,和VM的资源请求一样,代表一个独立的资源(即``箱子'',一般取值为$1$)。我们的目标是尽可能的减少``箱子''的数量并且能够满足将所有的VM请求部署在PM资源上且保证每个PM的上某一维上的请求都没有超过``箱子''的容量。因此,资源分配问题就可以视为一个d维装箱问题。




\section{问题定义}
\label{sec:prob-definition}

我们现在将所述问题进行数学定义:
\begin{defn}
\textbf{Vector Bin Packing problem}(VBP)


\label{VBP}

{\kai 给定一个有$n$个元素的$d$维向量的集合$S$,$S = \{\thinspace p_1,p_2,\cdots,p_n\ |\ p_i \in [0,1]^d \}$,寻找一个S上的划分$A_1,A_2,\cdots,A_m$使得$\sum_{p \in A_i}p^k\leq 1,\forall i,k$}
\end{defn}

VBP\ref{VBP}问题已经被证明是一个NP-Hard问题\cite{Garey76}。


\section{动机}
\label{sec:concern}


在很多企业提供的VPS(Virtual Private Server,虚拟专用服务器)服务中,提供给用户的选择很多都是不同的处理器性能和内存的大小。在实际的虚拟机部署中,除了资源分配的算法问题之外,还需要考虑如网络拓扑设计等许多方面的问题。但其中最关键也是最核心的问题还是如何根据用户的请求部署、移动虚拟机。在虚拟机部署考虑的资源中,计算资源和内部存储资源是最关键的,而外部存储资源(硬盘)等由于其成本相对较低,使用相对不频繁,因此通常并不做主要因素。

本文主要关注在理想环境下仅考虑计算资源(处理器核心)和内存的需求进行虚拟机部署的算法研究。



 






